\documentclass[a4paper,11pt]{article}
\usepackage[utf8]{inputenc}
\usepackage[greek,english]{babel}
\usepackage{alphabeta}
\usepackage{amsmath, amssymb, amsthm}
\usepackage{geometry}
\geometry{top=2cm, bottom=2cm, left=2cm, right=2cm}
\usepackage{fancyhdr}

% Header configuration
\pagestyle{fancy}
\fancyhf{}
\rhead{Ημερομηνία: \today}
\lhead{Μάθημα: Μαθηματικά}
\cfoot{\thepage}

\title{Διαγώνισμα Περιόδου}
\author{Όνομα Καθηγητή}
\date{\today}

\begin{document}

\begin{center}
    \Large \textbf{ΔΙΑΓΩΝΙΣΜΑ ΣΤΑ ΜΑΘΗΜΑΤΙΚΑ} \\
    \large Τάξη: ...
\end{center}
\vspace{1cm}

\section*{ΘΕΜΑ Α}
\textbf{A1.} Να αποδείξετε ότι... \hfill (Μονάδες 10) \\
\textbf{A2.} Να διατυπώσετε τον ορισμό... \hfill (Μονάδες 5) \\
\textbf{A3.} Σωστό ή Λάθος:
\begin{enumerate}
    \item ...
    \item ...
\end{enumerate} \hfill (Μονάδες 10)

\section*{ΘΕΜΑ Β}
Δίνεται η συνάρτηση $f(x) = ...$. 
\begin{enumerate}
    \item Να βρείτε το πεδίο ορισμού. \hfill (Μονάδες 8)
    \item ... \hfill (Μονάδες 9)
    \item ... \hfill (Μονάδες 8)
\end{enumerate}

\section*{ΘΕΜΑ Γ}
...

\section*{ΘΕΜΑ Δ}
...

\end{document}
