\documentclass[a4paper,11pt]{report}
\usepackage[utf8]{inputenc}
\usepackage[greek,english]{babel}
\usepackage{alphabeta}
\usepackage{amsmath, amssymb, amsthm}
\usepackage{geometry}
\geometry{top=2.5cm, bottom=2.5cm, left=2.5cm, right=2.5cm}
\usepackage{hyperref}
\usepackage{xcolor}
\usepackage{tcolorbox}

% Environments
\newtheorem{theorem}{Θεώρημα}[chapter]
\newtheorem{definition}{Ορισμός}[chapter]
\newtheorem{example}{Παράδειγμα}[chapter]
\newtheorem{exercise}{Άσκηση}[chapter]

\title{Σημειώσεις Μαθήματος}
\author{Συγγραφέας}
\date{\today}

\begin{document}
\maketitle
\tableofcontents

\chapter{Εισαγωγή}
\section{Βασικές Έννοιες}
Το μαθηματικό υπόβαθρο...

\begin{definition}
    Έστω $A$ ένα σύνολο...
\end{definition}

\section{Κύριο Μέρος}
\begin{theorem}
    (Θεώρημα Μέσης Τιμής) Αν μια συνάρτηση $f$...
\end{theorem}

\begin{proof}
    Έστω συνάρτηση $\phi(x)$...
\end{proof}

\begin{tcolorbox}[colback=blue!5!white,colframe=blue!75!black,title=Σημαντική Σημείωση]
    Προσοχή στις προϋποθέσεις του θεωρήματος!
\end{tcolorbox}

\end{document}
