\documentclass[a4paper,11pt]{article}
\usepackage[utf8]{inputenc}
\usepackage[greek,english]{babel}
\usepackage{alphabeta}
\usepackage{amsmath, amssymb, amsthm}
\usepackage{geometry}
\geometry{top=2.5cm, bottom=2.5cm, left=2.5cm, right=2.5cm}

\newtheorem{theorem}{Θεώρημα}
\newtheorem{lemma}[theorem]{Λήμμα}

\title{Τίτλος Επιστημονικού Άρθρου}
\author{Όνομα Συγγραφέα \\ \texttt{email@example.com}}
\date{\today}

\begin{document}

\maketitle

\begin{abstract}
    Εδώ γράφουμε την περίληψη του άρθρου (Abstract). Περιγράφουμε συνοπτικά το πρόβλημα, τη μέθοδο και τα βασικά αποτελέσματα.
\end{abstract}

\section{Εισαγωγή}
Η εισαγωγή θέτει το πλαίσιο του προβλήματος...

\section{Κύριο Μέρος}
Εδώ αναπτύσσεται η θεωρία και η μεθοδολογία.

\begin{theorem}
    Έστω $X$ ένας χώρος...
\end{theorem}

\section{Πειραματικά Αποτελέσματα}
Παρουσίαση δεδομένων, πινάκων και γραφικών παραστάσεων.

\section{Συμπεράσματα}
Σύνοψη των ευρημάτων και προτάσεις για μελλοντική έρευνα.

\begin{thebibliography}{9}
\bibitem{knuthwebsite} 
Knuth: Computers and Typesetting,
\\\texttt{http://www-cs-faculty.stanford.edu/\~{}uno/abcde.html}
\end{thebibliography}

\end{document}
